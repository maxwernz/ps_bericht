\documentclass[11pt,a4paper]{report} 
% dasselbe Template wie Thesis mit nur leichten Anpassungen
% Nehmen Sie das Thesis-Template für die Thesis!
% Lesen Sie Hinweise zum Umgang mit LaTeX und zum Schreiben
% von Berichten im Thesis-Template nach
% => Moodle => PraxissemesterThesis => LaTeXThesis.zip
%    https://moodle.hs-mannheim.de/course/view.php?id=2500


% Für doppelseitigen Ausdruck (nur bei > 60 Seiten sinnvoll)
% \usepackage{ifthen}
% \setboolean{@twoside}{true}
% \setboolean{@openright}{true} 

\include{preamble} % alle Pakete und Einstellungen

% Hier anpassen 
\newcommand{\autor}{Maximilian Wernz	}
\newcommand{\matrikelnummer}{2112041}
\newcommand{\fachsemester}{5} % im wie vielten Semester waren Sie?
\newcommand{\studiengang}{Medizintechnik}
%\newcommand{\studiengang}{Technische Informatik}
%\newcommand{\studiengang}{Informationstechnik}
\newcommand{\firma}{KLS Martin GmbH + Co. KG}
\newcommand{\standort}{Freiburg im Breisgau}
\newcommand{\abteilung}{Softwareentwicklung}
\newcommand{\betreuer}{Andreas Hagmüller}
\newcommand{\pbeginn}{01.03.2023}
\newcommand{\pende}{31.08.2023}
\newcommand{\tage}{123} % arbeitstagerechner verwenden!
\newcommand{\titel}{Bericht zum praktischen Studiensemester}
\newcommand{\kurztitel}{Praxisbericht}
% \sperrvermerktrue % Kommentar am Anfang der Zeile löschen für Sperrvermerk

\begin{document}
\begin{titlepage}
  \hsmalogo[1] \hfill
  \parbox[b]{60mm}{
    Fakultät Informationstechnik\\
    Studiengang \studiengang}
  \begin{center}
    \rule{1\textwidth}{1pt}\\[-3mm]
    \parbox[t][64mm]{110mm}{% 11 cm für Breite 13, ca. 7 für Höhe 6
      \Large{\ } \\[8mm]
      Bericht zum praktischen Studiensemester \\[4mm]
      \begin{tabular}{rl}
        \large{Vorgelegt von} & \large{\autor} \\[2mm]
        \large{Studiengang} & \large{\studiengang} \\[2mm]
        \large{Firma} & \large{\firma} \\[2mm]
      \end{tabular}
    }
    \rule{\textwidth}{1pt}    
    \vfill    
  \end{center}

  \vspace{2em}
  \begin{tabular}{ll}
    Name & \autor \\
    Matrikelnummer & \matrikelnummer \\
    Studiengang & \studiengang \\
    Fachsemester & \fachsemester \\[8mm]

    Praktikumszeitraum & \pbeginn\ \ bis \ \pende \\
    Präsenztage & \tage \\[8mm]
    
    Firma & \firma \\
    Standort & \standort \\
    Abteilung & \abteilung \\
    Betreuer & \betreuer \\
  \end{tabular}
  
  \vspace{8em}
  \noindent\begin{tabular}{p{0.48\textwidth}p{0.48\textwidth}}
             \rule{0.46\textwidth}{0.5pt} & \rule{0.46\textwidth}{0.5pt}  \\
             Datum, \betreuer & Firmenstempel
           \end{tabular}
  \vfill
\end{titlepage}
\cleardoublepage


% Erklärung gemäß der Prüfungsordnung
\thispagestyle{empty}
\subsection*{Selbstständigkeitserklärung}

Ich versichere, dass ich diesen Bericht zum praktischen Studiensemester
selbstständig und nur unter Verwendung der angegebenen Quellen und
Hilfsmittel angefertigt habe.
Die Stellen, an denen Inhalte aus den Quellen verwendet wurden, sind
als solche eindeutig gekennzeichnet.
Die Arbeit hat in gleicher oder ähnlicher Form bei keinem anderen
Prüfungsverfahren vorgelegen.

\vspace{6em}
\noindent\begin{tabular}{p{0.48\textwidth}p{0.48\textwidth}}
\rule{0.42\textwidth}{0.5pt} & \rule{0.48\textwidth}{0.5pt}\\
Datum, Ort                   & \makebox[1cm]{\ } \autor
\end{tabular}

\vfill

\ifsperrvermerk
\subsection*{Sperrvermerk}

Der vorliegende Bericht enthält interne und teilweise vertrauliche Daten
der Firma \newline \firma.
Der Bericht darf daher zu keinen anderen als Prüfungszwecken verwendet werden.
Insbesondere ist die Vervielfältigung und Veröffentlichung von Berichtinhalten
oder Teilen davon nur mit Zustimmung des Unternehmens erlaubt.
\vfill
\fi

\cleardoublepage

 % Titelseite, Erklärungen, etc.

\begin{abstract}
  Beim führenden Anbieter von Fallen aller Art ist die
  Produktentwicklung geprägt von Kreativität und höchsten
  Qualitätsansprüchen, die schon in der Forschungs- und
  Entwicklungsabteilung umfassende Tests und Messungen
  nach sich zieht.
  
  Bei ausgiebigen Materialprüfungen konnte bei Federn für
  Steingewichte $>5$ Tonnen eine ungewünschte Zielabweichung
  bestätigt werden. Beim Aufbau der Prüfumgebung, insbesondere der
  Nachverfolgung der Flugbahn mit Kameras wurde die Genauigkeit
  der Messung mit kleineren Steinen zunächst bestimmt und dann
  die Abweichung in Abhängigkeit der Masse belegt.
  Eine deutliche Verbesserung von 5 Sekunden auf 0,1 Sekunden
  konnte bei der Auslösung der Raketentriebwerke erreicht werden.
  Durch die massive Zunahme der Anzahl der Sensormessungen war
  das eingesetzte Sortierverfahren nicht mehr in der Lage die
  Berechnungen rechtzeitig abzuschließen.
  Durch den Einsatz eines effizienten, stabilen Standardsortierverfahrens
  ist dieser Teil des Gesamtaufwands bis zu 10000 Sensorwerten von bis
  zu 5 Sekunden auf immer unter 10 ms auf dem eingesetzten Mikrocontroller
  ACME (ACME Controlled Micro Engine) 503 geschrumpft.
  Auch bei den bisher mechanisch ausgeführten Langwaffen (Bogen) konnte
  durch den Einsatz von ausgeklügelter Elektronik die Anwendbarkeit
  deutlich verbessert werden.
  Das entwickelte System meldet gegebenenfalls einen 416-Fehler, falls
  das Ziel für einen erfolgreichen Einsatz zu weit entfernt ist.
  Ein versuchter Abschuss bei eingerichteter Sperre, der vorher zu
  erheblichen Schäden am Gerät geführt hat, ergibt jetzt einen
  benutzerfreundlichen 423-Fehler.
  Falls das eingesetzte Geschoss (Chip-kodiert) nicht für die
  Abschussgeschwindigkeit freigegeben ist oder die Zielführung kein
  relevantes Ziel verifizieren kann, wird dies mit einem
  428-Fehler angezeigt.
\end{abstract}

\tableofcontents

\chapter{Einführung} \label{chap:einf}

Die Firma
% \gls{acme} % das fügt ein Glossar-Eintrag ein
ACME Inc.\cite{acme,acmecatalog,kenner1994chuck} ist internationaler Marktführer
bei Fallen aller Art.
In der deutschen Niederlassung ist neben dem Vertrieb eine
Entwicklungsabteilung ansässig, um Qualitätsmängel bei frühen
Produktversionen zu identifizieren und gegebenenfalls abzustellen.
Bei der Produktisierung sollte der gesamte Lebenszyklus,
wie in Abbildung~\ref{fig:lifecycle} gezeigt,
\begin{figure}[htp]
  \centering
  % \includegraphics[width=.9\textwidth]{ablauf}
  \caption{Lebenszyklus eines Produkts von ACME Inc.}
  \label{fig:lifecycle}
\end{figure}
berücksichtigt werden.\marginpar{Fokus auf Ihren Aufgaben}
Ziel ist dabei immer den Umsatz in einem Segment aus Tabelle~\ref{tab:umsatz}
zu steigern.
\begin{table}[htp]
  \centering
  \begin{tabular}{l|rrrrrrrrrr}
    Fallentyp/Jahre    & 56 & 57 & 58 & 59 & 60 & 61 & 62 & 63 & 64 & 65 \\\hline\hline
    Mechanische Fallen & 1  & 3  & 5  & 8  & 8  &  8 &  5 &  5 & 10 & 7 \\
    Raketenfallen      & -  & -  & -  & 2  & 3  &  6 &  7 &  9 &  3 & 2 \\
    Bogen              & 2  & 3  & 2  & 2  & 3  &  4 &  3 &  2 &  2 & 3 \\
  \end{tabular}
  \caption{Umsatz-Segmente von ACME Inc. von 1956-1965, in Tausend Dollar}
  \label{tab:umsatz}
\end{table}
Langfristig sollte dann aber meist auch die Funktionalität garantiert
werden können.
Im Praktikum \ldots


%% > weg damit
\vskip\medskipamount % or other desired dimension
\leaders\vrule width \textwidth\vskip0.4pt
\vskip\medskipamount % ditto
\nointerlineskip

Bitte lesen Sie die Informationen zum Praxissemester~\cite{psmoodle}
der Fakultät in Moodle.
Sie halten sich an die Vorgaben im
„Leitfaden zur Berichterstellung“~\cite{psberichtleitfaden},
der auch in Moodle erhältlich ist.
Für Hinweise zu \LaTeX~\cite{kopka,lamport,bibtexing,knuth} und zum
Anfertigen einer Ausarbeitung~\cite{gockel,rechenberg} lesen Sie bitte
auch das Thesis-Template in der
Moodle Gruppe \texttt{PraxissemesterThesis}~\cite{psmoodle}.
Ernsthaft, lesen Sie das Thesis-Template.
Da stehen wirklich Hinweise zum Schreiben.
Sie sollen dann durchaus viele Abbildungen und gegebenenfalls auch
Tabellen in Ihren Text einbringen und nicht, wie hier im Template,
nur Textwüsten.

Die Vorstellung der Firma\marginpar{Vorstellung der Firma}
braucht meist kein eigenes Kapitel und auf keinen Fall ein Logo.
Sie machen selten eine Analyse der Firmenstruktur und die
historische Entwicklung der Firma macht nur Seiten, aber hilft
inhaltlich nicht.
Also schreiben Sie in der Einleitung maximal nur einen Absatz
zu der Firma und der Abteilung und konzentrieren sich dann auch
die Inhalte.
Wir schließen die Einleitung mit einer inhaltlichen Übersicht
über die folgenden Kapitel.

In Kapitel~\ref{chap:ppv} stellen wir die drei typischen Fallenprodukte
vor sowie die passenden Prüfverfahren zur Verbesserung der
mechanischen Genauigkeit und zur Reduktion der zeitlichen Verzögerung.
In Kapitel~\ref{chap:sling} erläutern wir den Aufbau und das Messverfahren
für die Abweichung von Schleudern beim Einsatz von schweren Gewichten
über mehrere Tonnen.
In Kapitel~\ref{chap:rocket} besprechen wir die erzielte Auslösereaktion
durch Einsatz adäquater Sortierverfahren auf dem Mikrocontroller für die
Raketensteuerung.
Die Verbesserungen der Benutzbarkeit von Bögen unter Einsatz moderner Technik
sammeln wir in Kapitel~\ref{chap:bow}.
Abschließend fassen wir in Kapitel~\ref{chap:fazit} die im Praktikum
erzielten mechanischen und zeitlichen Verbesserungen sowie die
erhöhte Benutzbarkeit noch einmal zusammen.


\chapter{Produkte und Prüfverfahren} \label{chap:ppv}

Trotz Fokus auf den Anwendungsbereich Fallen wird ein weiter Bereich
von technischer Infrastruktur im Produktangebot abgedeckt.
Die eingesetzten Prüfverfahren testen meist eine mechanische
und zeitliche Genauigkeit.

\section{Schleudern, Raketen und Bögen} \label{sec:was}

\blindtext[2]
\blindtext[2]

\section{Mechanische Genauigkeit} \label{sec:mec}

\blindtext[2]
\blindtext[1]
\blindtext[2]

\section{Zeitliche Abweichungen} \label{sec:time}

\blindtext[1]
\blindtext[2]
\blindtext[1]

\chapter{Genauigkeit von Schleudern} \label{chap:sling}

Die Genauigkeit von Schleudern hängt von den beiden Faktoren
Eins und Zwei ab und wird mit Drei oder Vier gemessen.
Eins kombiniert X und Y, um qualitative Aussagen über den
strukturellen Aufbau zu machen.
Im Gegensatz dazu setzt sich Zwei aus A, B und C zusammen
und erlaubt eine quantitative Festlegung.
Das Verfahren Drei liefert mit wenig Aufwand einen ersten
Anhaltspunkt, ob weitere detaillierte Messungen notwendig sind.
Mit dem aufwendigen Verfahren Vier können wir dann Toleranzen
auf $\pm 0.1\%$ Abweichung bestimmen.

\section{Eins}

\blindtext[3]
\blindtext[1]
\blindtext[2]
\blindtext[3]

\section{Zwei}

\blindtext[1]
\blindtext[2]
\blindtext[4]
\blindtext[2]

\section{Drei}

\blindtext[3]
\blindtext[2]
\blindtext[2]
\blindtext[1]

\chapter{Steuern von Raketentriebwerken} \label{chap:rocket}

Raketentriebwerke werden hauptsächlich mechanisch gesteuert.
Nur das Anstoßen von Zustandsübergängen Zwei wird elektronisch
getriggert. Drei ist eine übliche Realisierung.

\section{Eins}

\blindtext[2]
\blindtext[4]
\blindtext[3]
\blindtext[3]
\blindtext[2]

\section{Zwei}

\blindtext[3]
\blindtext[1]
\blindtext[3]
\blindtext[4]
\blindtext[1]

\section{Drei}

\blindtext[2]
\blindtext[3]
\blindtext[2]
\blindtext[4]

\chapter{Benutzbarkeit von Bögen} \label{chap:bow}

Text\ldots

\section{Eins}

\blindtext[1]
\blindtext[2]
\blindtext[3]
\blindtext[2]
\blindtext[1]

\section{Zwei}

\blindtext[2]
\blindtext[2]
\blindtext[3]
\blindtext[2]

\section{Drei}

\blindtext[3]
\blindtext[5]
\blindtext[3]
\blindtext[2]


\chapter{Fazit} \label{chap:fazit}

\blindtext[2]
\blindtext[1]


\newpage

% Listen wenn überhaupt ans Ende und nicht an den Anfang.
% Meist ist das aber unnötig.
%\listoffigures % Liste der Abbildungen 
%\begingroup % aahh nicht noch ein pagebreak
%\let\clearpage\relax %
%\listoftables % Liste der Tabellen
%\endgroup

% Glossar kommt auch ans Ende
%\glsaddall % das fügt alle Glossar-Einträge ein
%\printglossaries % nicht vergessen "makeglossaries praksem" aufzurufen
%\newpage

\addcontentsline{toc}{chapter}{Literaturverzeichnis}
\bibliographystyle{plain} % Literaturverzeichnis
\bibliography{praksem}
% \bibliography{praksem,online} # wenn man zwei Dateien hätte

% Das wäre die Alternative mit geteilten Quellen (preamble muss auch
% angepasst werden) und die Literatur muss in die Datei praksem.bib
% und die Online-Quellen müssen in die Datei online.bib.
%\begin{btSect}{praksem} % mit bibtopic Quellen trennen
%\section*{Literaturverzeichnis}
%\addcontentsline{toc}{chapter}{Literaturverzeichnis}
%\btPrintCited
%\end{btSect}
%\begin{btSect}{online}
%\section*{Online-Quellen}
%\addcontentsline{toc}{chapter}{Online-Quellen}
%\btPrintCited
%\end{btSect}
% dann ab und zu "bibtex praksem1" und "bibtex praksem2" aufrufen

\end{document}
;;; Local Variables:
;;; ispell-local-dictionary: "de_DE-neu"
;;; End:
