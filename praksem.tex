\documentclass[11pt,a4paper]{report} 

% Deutsch
\usepackage[german]{babel} % deutsch und deutsche Rechtschreibung
\usepackage[utf8]{inputenc} % Unicode Text 
\usepackage[T1]{fontenc} % Umlaute und deutsches Trennen
\usepackage{textcomp} % Euro
\usepackage[hyphens]{url}
% statt immer Ab\-schluss\-ar\-beit zu schreiben
% einfach hier sammeln mit -. 
\hyphenation{Ab-schluss-ar-beit}
% Vorsicht bei Umlauten und Bindestrichen
\hyphenation{Ver-st\"ar-ker-aus-gang}
 % eigene Hyphenations, die für das Dokument gelten
\usepackage{amssymb} % Symbole
\usepackage{emptypage} % Wirklich leer bei leeren Seiten

%% Fonts, je ein kompletter Satz an Optionen

% Times New Roman, gewohnter Font, ok tt und serifenlos
%\usepackage{mathptmx} 
%\usepackage[scaled=.95]{helvet}
%\usepackage{courier}

% Palatino mit guten Fonts für tt und serifenlos
\usepackage{mathpazo} % Palatino, mal was anderes
\usepackage[scaled=.95]{helvet}
\usepackage{courier}

% New Century Schoolbook sieht auch nett aus (macht auch tt und serifenlos)
%\usepackage{newcent}

% Oder default serifenlos mit Helvetica 
% ich kann es nicht mehr sehen ...
%\renewcommand{\familydefault}{\sfdefault}

% ein bisschen eine bessere Verteilung der Buchstaben...
\usepackage{microtype}

% Bilder und Listings
\usepackage{graphicx} % wir wollen Bilder einfügen
\usepackage{subfig} % Teilbilder
\usepackage{wrapfig} % vielleicht doch besser vermeiden
\usepackage{listings} % schöne Quellcode-Listings
% ein paar Einstellungen für akzeptable Listings
\lstset{basicstyle=\ttfamily, columns=[l]flexible, mathescape=true, showstringspaces=false, numbers=left, numberstyle=\tiny}
\lstset{language=python} % und nur schöne Programmiersprachen ;-)
% und eine eigene Umgebung für Listings
\usepackage{float}
\newfloat{listing}{htbp}{scl}[chapter]
\floatname{listing}{Listing}

% Seitenlayout
\usepackage[paper=a4paper,width=14.8cm,left=30mm,right=30mm,height=23cm]{geometry}
\usepackage{setspace}
\linespread{1.15}
\setlength{\parskip}{0.5em}
\setlength{\parindent}{0em} % im Deutschen Einrückung nicht üblich, leider

% Seitenmarkierungen 
\newcommand{\phv}{\fontfamily{phv}\fontseries{m}\fontsize{9}{11}\selectfont}
\usepackage{fancyhdr} % Schickere Header und Footer
\pagestyle{fancy}
\renewcommand{\chaptermark}[1]{\markboth{#1}{}}
%\fancyhead[L]{\phv \leftmark}
\fancyhead[RE,LO]{\phv \nouppercase{\leftmark}}
\fancyhead[LE,RO]{\phv \thepage}
% Unten besser auf alles Verzichten
%\fancyfoot[L]{\textsf{\small \kurztitel}}
\fancyfoot[C]{\ } % keine Seitenzahl unten
%\fancyfoot[R]{\textsf{\small Technische Informatik}}

% Theorem-Umgebungen
\newtheorem{definition}{Definition}[chapter]
\newtheorem{satz}{Satz}[chapter]
\newtheorem{lemma}[satz]{Lemma} % gleicher Zähler wie Satz
\newtheorem{theorem}{Theorem}[chapter]
\newenvironment{beweis}[1][Beweis]{\begin{trivlist}
\item[\hskip \labelsep {\textit{#1 }}]}{\end{trivlist}}
\newcommand{\qed}{\hfill \ensuremath{\square}}

%% Quellen
% Eine Alternative wäre Quellen in Literatur und Online-Quellen
% zu teilen
% \usepackage{bibtopic} 

% Hochschule Logo, noch nicht perfekt
\usepackage{hsmalogo}

% Spezialpakete
\usepackage{epigraph}
\setlength{\epigraphrule}{0pt} % kein Trennstrich

% damit wir nicht so viel tippen müssen, nur für Demo 
\usepackage{blindtext} 

% ifthen für sperrvermerk
\newif\ifsperrvermerk
 % alle Pakete und Einstellungen

% Hier anpassen 
\newcommand{\autor}{Maximilian Wernz}
\newcommand{\matrikelnummer}{2112041}
\newcommand{\fachsemester}{5} % im wie vielten Semester waren Sie?
\newcommand{\studiengang}{Medizintechnik}
%\newcommand{\studiengang}{Technische Informatik}
%\newcommand{\studiengang}{Informationstechnik}
\newcommand{\firma}{KLS Martin GmbH + Co. KG}
\newcommand{\standort}{Freiburg im Breisgau}
\newcommand{\abteilung}{Softwareentwicklung}
\newcommand{\betreuer}{Andreas Hagmüller}
\newcommand{\pbeginn}{01.03.2023}
\newcommand{\pende}{31.08.2023}
\newcommand{\tage}{123} % arbeitstagerechner verwenden!
\newcommand{\titel}{Bericht zum praktischen Studiensemester}
\newcommand{\kurztitel}{Praxisbericht}
% \sperrvermerktrue % Kommentar am Anfang der Zeile löschen für Sperrvermerk

% Wenn jemand unbedingt ein Glossar will, die nächsten drei Zeilen...
%\usepackage{glossaries} % oder schlimmer mit [toc], damit es im TOC auftaucht
%\makeglossaries
%\newglossaryentry{Computer}{name=Computer,
  description={Eine programmierbare Maschine, die Eingaben erhält, Daten speichert und manipuliert
    und Ausgaben in einem sinnvollem Format ausgibt. (Und wer so was in ein Glossar eines Berichts
    für einen technischen Studiengang schreibt hat es nicht verstanden)}}
\newglossaryentry{naiv}{name=na\"{\i}ve,
  description={Ein franzöisches Lehnswort (Adjektiv, Form von naïf)
    Erweckt den Eindruch oder hat mangelnde Erfahrung, mangelndes Verständnis oder mangelnde Skills}}
\newglossaryentry{Linux}{name=Linux,
  description={Generischer Ausdruck für eine Familie von Unix-artigen Betriebssystemen die den
    Linux-Kernel verwenden},
  plural=Linuces}
\newacronym[longplural={Frames per Second}]{fpsLabel}{FPS}{Frame per Second}
\newacronym{acme}{ACME}{A Company Making Everything}
\newglossaryentry{Praktisches Studiensemester}{name=Praktisches Studiensemester,
  description={Im Rahmen des Ingenieursstudiums ein Semester in der betrieblichen Praxis
    zur Ergänzung und Vertiefung des Studienwissens durch selbstständige ingenieurnahe Tätigkeit
    betreut durch einen Ingenieur des Betriebes}}
 % In dieser Datei die Einträge definieren
% und noch ganz unten printglossaries auskommentieren
% Damit jetzt ein Glossar gezeigt wird noch \gls{label} verwenden

\begin{document}
\begin{titlepage}
  \hsmalogo[1] \hfill
  \parbox[b]{60mm}{
    Fakultät Informationstechnik\\
    Studiengang \studiengang}
  \begin{center}
    \rule{1\textwidth}{1pt}\\[-3mm]
    \parbox[t][64mm]{110mm}{% 11 cm für Breite 13, ca. 7 für Höhe 6
      \Large{\ } \\[8mm]
      Bericht zum praktischen Studiensemester \\[4mm]
      \begin{tabular}{rl}
        \large{Vorgelegt von} & \large{\autor} \\[2mm]
        \large{Studiengang} & \large{\studiengang} \\[2mm]
        \large{Firma} & \large{\firma} \\[2mm]
      \end{tabular}
    }
    \rule{\textwidth}{1pt}    
    \vfill    
  \end{center}

  \vspace{2em}
  \begin{tabular}{ll}
    Name & \autor \\
    Matrikelnummer & \matrikelnummer \\
    Studiengang & \studiengang \\
    Fachsemester & \fachsemester \\[8mm]

    Praktikumszeitraum & \pbeginn\ \ bis \ \pende \\
    Präsenztage & \tage \\[8mm]
    
    Firma & \firma \\
    Standort & \standort \\
    Abteilung & \abteilung \\
    Betreuer & \betreuer \\
  \end{tabular}
  
  \vspace{8em}
  \noindent\begin{tabular}{p{0.48\textwidth}p{0.48\textwidth}}
             \rule{0.46\textwidth}{0.5pt} & \rule{0.46\textwidth}{0.5pt}  \\
             Datum, \betreuer & Firmenstempel
           \end{tabular}
  \vfill
\end{titlepage}
\cleardoublepage


% Erklärung gemäß der Prüfungsordnung
\thispagestyle{empty}
\subsection*{Selbstständigkeitserklärung}

Ich versichere, dass ich diesen Bericht zum praktischen Studiensemester
selbstständig und nur unter Verwendung der angegebenen Quellen und
Hilfsmittel angefertigt habe.
Die Stellen, an denen Inhalte aus den Quellen verwendet wurden, sind
als solche eindeutig gekennzeichnet.
Die Arbeit hat in gleicher oder ähnlicher Form bei keinem anderen
Prüfungsverfahren vorgelegen.

\vspace{6em}
\noindent\begin{tabular}{p{0.48\textwidth}p{0.48\textwidth}}
\rule{0.42\textwidth}{0.5pt} & \rule{0.48\textwidth}{0.5pt}\\
Datum, Ort                   & \makebox[1cm]{\ } \autor
\end{tabular}

\vfill

\ifsperrvermerk
\subsection*{Sperrvermerk}

Der vorliegende Bericht enthält interne und teilweise vertrauliche Daten
der Firma \newline \firma.
Der Bericht darf daher zu keinen anderen als Prüfungszwecken verwendet werden.
Insbesondere ist die Vervielfältigung und Veröffentlichung von Berichtinhalten
oder Teilen davon nur mit Zustimmung des Unternehmens erlaubt.
\vfill
\fi

\cleardoublepage

 % Titelseite, Erklärungen, etc.

\begin{abstract}

\end{abstract}

\tableofcontents

\chapter{Einführung} \label{chap:einf}

% \gls{acme} % das fügt ein Glossar-Eintrag ein

\chapter{Produkte und Prüfverfahren} \label{chap:ppv}

\section{Schleudern, Raketen und Bögen} \label{sec:was}

\section{Mechanische Genauigkeit} \label{sec:mec}

\section{Zeitliche Abweichungen} \label{sec:time}

\chapter{Genauigkeit von Schleudern} \label{chap:sling}

\section{Eins}

\section{Zwei}

\section{Drei}

\chapter{Steuern von Raketentriebwerken} \label{chap:rocket}

\section{Eins}

\section{Zwei}

\section{Drei}

\chapter{Benutzbarkeit von Bögen} \label{chap:bow}

\section{Eins}

\section{Zwei}

\section{Drei}



\chapter{Fazit} \label{chap:fazit}


\newpage

% Listen wenn überhaupt ans Ende und nicht an den Anfang.
% Meist ist das aber unnötig.
%\listoffigures % Liste der Abbildungen 
%\begingroup % aahh nicht noch ein pagebreak
%\let\clearpage\relax %
%\listoftables % Liste der Tabellen
%\endgroup

% Glossar kommt auch ans Ende
%\glsaddall % das fügt alle Glossar-Einträge ein
%\printglossaries % nicht vergessen "makeglossaries praksem" aufzurufen
%\newpage

\addcontentsline{toc}{chapter}{Literaturverzeichnis}
\bibliographystyle{plain} % Literaturverzeichnis
\bibliography{praksem}
% \bibliography{praksem,online} # wenn man zwei Dateien hätte

% Das wäre die Alternative mit geteilten Quellen (preamble muss auch
% angepasst werden) und die Literatur muss in die Datei praksem.bib
% und die Online-Quellen müssen in die Datei online.bib.
%\begin{btSect}{praksem} % mit bibtopic Quellen trennen
%\section*{Literaturverzeichnis}
%\addcontentsline{toc}{chapter}{Literaturverzeichnis}
%\btPrintCited
%\end{btSect}
%\begin{btSect}{online}
%\section*{Online-Quellen}
%\addcontentsline{toc}{chapter}{Online-Quellen}
%\btPrintCited
%\end{btSect}
% dann ab und zu "bibtex praksem1" und "bibtex praksem2" aufrufen

\end{document}
;;; Local Variables:
;;; ispell-local-dictionary: "de_DE-neu"
;;; End:
